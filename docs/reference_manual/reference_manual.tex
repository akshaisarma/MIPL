% acmsmall-sample.tex, dated 4th Nov. 2011
% This is a sample file for ACM small trim journals
%
% Compilation using 'acmsmall.cls' - version 1.2, Aptara Inc.
% (c) 2010 Association for Computing Machinery (ACM)
%
% Questions/Suggestions/Feedback should be addressed to => "acmtexsupport@aptaracorp.com".
% Users can also go through the FAQs available on the journal's submission webpage.
%
% Steps to compile: latex, bibtex, latex latex
%
% For tracking purposes => this is v1.2 - Nov. 2011

\documentclass[prodmode,acmtecs]{acmsmall}

% Package to generate and customize Algorithm as per ACM style
\usepackage[ruled]{algorithm2e}
\usepackage{listings}
\usepackage{longtable}
\renewcommand{\algorithmcfname}{ALGORITHM}
\SetAlFnt{\large}
\SetAlCapFnt{\large}
\SetAlCapNameFnt{\large}
\SetAlCapHSkip{10pt}
\IncMargin{-\parindent}

% Metadata Information
\acmVolume{1}
\acmNumber{1}
\acmArticle{1}
\acmYear{2012}
\acmMonth{3}

% Document starts
\begin{document}

% Page heads
%\markboth{Young Hoon et al.}{MIPL Reference Manual}

% Title portion
\title{{MIPL: Mining Integrated Programming Language\\
Language Reference Manual}}
\author{Project Manager: Younghoon Jeon $<yj2231@columbia.edu>$\\
System Architect: Young Hoon Jung $<yj2244@columbia.edu>$\\
Language Expert: Jinhyung Park $<jp2105@columbia.edu>$\\
System Integrator: Daniel Wonjoon Song $<dws2127@columbia.edu>$\\
System Validation: Akshai Sarma $<as4107@columbia.edu>$
%\affil{Columbia University}
}

\begin{abstract}
MIPL is a language for Data Mining which provides methods to 
implement various sophisticated matrix-based algorithms and 
to retrieve the computed data easily by offering a novel,
convenient way that converts matrices into a set of Facts
and vice versa.
\end{abstract}

%\category{C.2.2}{Computer-Communication Networks}{Network Protocols}

%\terms{Design, Algorithms, Performance}

%\keywords{Wireless sensor networks, media access control,
%multi-channel, radio interference, time synchronization}

%\acmformat{Zhou, G., Wu, Y., Yan, T., He, T., Huang, C., Stankovic,
%J. A., and Abdelzaher, T. F.  2010. A multifrequency MAC specially
%designed for  wireless sensor network applications.}

%\begin{bottomstuff}
%This work is supported by the National Science Foundation, under
%\end{bottomstuff}

\maketitle

\tableofcontents 

\pagebreak

\lstset{frame=single}

\section{Introduction}

Mining Integrated Programming Language (MIPL) is a programming language
that provides a very simple, easy way for the users to use Data Mining
algorithms and retrieve data from their computations. At the same time, 
MIPL also enables developers to easily design these algorithms with 
explicit support for working with large matrices and matrix operations.

Algorithm developers work with MIPL's Job interface, where MIPL allows
for programming constructs, such as conditionals and iterations, with
support for working with matrices. A Job can be likened to a traditional
function but focused toward MIPL's target audience: data miners. The JOB 
interface sports novel concepts such as returning more than one value.
Furthermore, algorithms developed as a Job are automatically distributed 
and accelerated on the computer clusters and hardware accelerators through
MapReduce. 

Regular users, who are perhaps, only interested in obtaining results from
running a Data Mining Job, work with MIPL's FRQ interface, which allows
the user to declare Facts, Rules and make Queries. Regular expressions are
supported, enabling a user to write powerful, elegant and concise FRQ's 
that can use Job to compute and produce results to queries. A key 
concept in MIPL is that Facts in the FRQ can be converted to matrices
in Job and vice versa, connecting the simple interface of the FRQ with
Job.

\medskip

% Head 1
\section{Lexical conventions}
MIPL has six kinds of tokens: identifiers, variables, numbers, strings,
expression operators, and other separators.  In general blanks, tabs,
newlines, and comments as described below are ignored except to
serve as token seperators. At least one of these characters is required
to separate otherwise adjacent identifiers, constants, and certain
operatorpairs.  If the input stream has been parsed into tokens up
to a given character, the next token is taken to include the longest
string of characters which could possibly constitute a token.

In MIPL, an identifiers is a word that starts with a lower case alphabet
and a variable is a word that starts with a upper case alphabet.  For
example, cat is a indentifier and Name is a variable.
\medskip

% Head 2
\subsection{Comments}

MIPL uses shell script-like comments, which begins with the character
\# and terminating at the end of the line, denoted by the newline character.
\medskip

% Head 3
\subsection{Identifiers}

An Identifier is a sequence of letters that begins with lower case alphabets.
Upper and lower case letters are considered different.   Identifiers are
served for use as the name of Terms, the basic block of MIPL syntax.
\medskip

\subsection{Variables}

A Variable is a sequence of letters that begins with upper case alphabets
or the underscore character, \_, which means that this variable will not
be tracked.  Upper and lower case letters in the variable names are
considered different. A Variable is used for convenience sake, to distinguish
it from an identifier. A Variable always has a location value, while identifiers
can refer to Facts in the FRQ or bindings.
\medskip

\subsection{Numbers}

A number in MIPL can be integeres or floating point numbers.  A number consists of
an integer part, a decimal point, a fractional part, an e or E, and a signed 
integer exponent. The integer and fraction parts both consist of sequences of decimal
digits.
Either the integer part or the fraction part (not both) may be missing; either 
the decimal point or the e or E and the exponent (not both) may be missing.
Every floating constant is taken at double precision.
\medskip

\subsection{Strings}

A string is a sequence of characters surrounded by double quotes or single
quotes.
\medskip

\subsection{Expression Operators}
Expression Operators are explained in detail Built-in Operators and Jobs section.
\medskip

\section{Data Types}

\subsection {Term}
Term is the only data type for Facts, Rules, and Queries.  Numbers, Strings,
Facts, Rules, Queries and even Terms are recursively defined in a form of a Term. 
Variables in a Term can contain a Term and their scope is within the Command.
\medskip

\subsection {Job Types}
Unlike the FRQ part of MIPL, a Job has two primitive types, Matrix and
Number, which can both be computed within a Job description. MIPL strictly
distinguish boolean expressions from numeric expressions, so boolean can be
considered another primitive type, however it is not visible to the users
as there is no variables that can contain boolean values.  Variables in a
Job can contain a Number or a Matrix and their scope is within the Job
definition.

\section{Syntax}

%----------------------------------------------------------JINHYUNG{
\subsection{Program}

A MIPL program is simple. Basically, the MIPL program looks like below:

\begin{lstlisting}
program
	: commands
	;
\end{lstlisting}

In other words, a MIPL program consists of commands, where commands can contain multiple 
commmands again.

MIPL can run in an interpreter mode or can compile, much like Python or Prolog. A command
is the basic unit for MIPL and directly maps to a line in the interpreter mode in MIPL.
More specifically, when the user writes a command to the
input terminal, the command is executed right away by the interpreter,
whereas the compiler generates target machine code so that it can be
executed by running the generated target machine code.  However, the
results from the same MIPL program are always identical regardless of
whether it is executed by the interpreter or through compiled code.

Here is a sample MIPL program:

\begin{lstlisting}
job countPositive(A, SizeOfA) {
    I = 0.
    while (I < SizeOfA) {
        if (A[I] > 0) {
            I = I + 1.
        }
    }
}
cat(tom).
cat(tom)?
animal(X) <- cat(X).
animal(tom)?
\end{lstlisting}

In this example, there are five commands:
$countPositive\{\}$, $cat().$, $cat()?$, $animal() <- cat().$, and $animal()?$. Each
command can also include other commands. Each command can be identified by these:
$\{\}, .,$ or $?$. These are used to determine the command type.
\medskip

\subsection{Commands}

\begin{lstlisting}
commands
	: command commands
	| command
	;
\end{lstlisting}

Including other commands in a command is permissible. In other words, some commands
can contain as part of their definition, more commands. The previous example in 4.1
shows that $countPositive\{\}$ includes several other commands. 
\medskip

\subsection{Command}

In MIPL, there are four types of commands: $fact$, $query$, $rule$, and $job$.

\begin{lstlisting}
command
	: fact
	| query
	| rule
	| job
	;
\end{lstlisting}
\medskip


\subsubsection{Fact}
\begin{lstlisting}
fact
	: term `.'			
	| `[' id_list `]' LARROW_OP jobcall `.'
	;

id_list
	: id_list `,' IDENTIFIER
	| IDENTIFIER		
	;

jobcall
	: IDENTIFIER `(' `)'
	| IDENTIFIER `(' jobcall_args `)'
	;

jobcall_args
	: jobcall_args `,' jobcall_args_cand
	| jobcall_args_cand
	;

jobcall_args_cand
	: IDENTIFIER
	| VARIABLE
	| NUMBER
	| STRING_LITERAL
	| jobcall
	;
\end{lstlisting}

\begin{lstlisting}
# fact
cat(tom).
mouse(jerry), mouse(mickey).
\end{lstlisting}

A $fact$ can be Term, which is terminated with a $"."$. A $fact$ is used to
describe or define a Fact (which expresses some statement in the context
of the application) in MIPL, or to process a Job. There are two kinds of
Facts: static and dynamic Facts. A static Fact is a Term that is terminated
with $"."$. A dynamic Fact is a Term that is wrapped in $[ ]$, and as part
of its definition, a dynamic Fact calls a Job to process data. The semantics
of the $jobcall$ are exactly the same as the dicussion in \ref{Job_PostFix_Expression}
and it is not duplicated here. A $fact$ should start with a lowercase letter.
\medskip

\subsubsection{Query}
\begin{lstlisting}
query
	: or_terms `?'		
	;
\end{lstlisting}

\begin{lstlisting}
# fact
cat(tom), mouse(jerry).
# query
mouse(tom)? # Returns false
# query
mouse(X)? # Returns X = jerry
# fact
weighted_grade(86, 99, 32).
# query with *
weighted_grade(*)? # Returns 86, 99, 32
\end{lstlisting}

A $query$ is terminated with a $"?"$. A $query$ operates on an OR Term. A $query$ 
are applied to Facts and Rules. Even if there are no direct matched Facts, Rules
can be used to derive new Facts to determine the result of a Query. A $query$ to 
a Fact returns $true$ or $false$ if the Query contains only identifiers. A $query$ 
to a Fact returns binding results of Facts satisfying the Query if the Query contains 
Variables. If the Fact contains multiple  arguments, then the Query can contain $*$
as the query argument to show all bindings, regardless of the number of arguments, 
whereas using Queries with Variables will require specifying the same number of 
Variables as arguments of the Fact.
\medskip

\subsubsection{Rule}
\begin{lstlisting}
rule
	: term LARROW_OP or_terms `.'
	;
\end{lstlisting}

\begin{lstlisting}
# rule with a fact
animal(X) <- cat(X).
# another rule operator
animal(X) :- cat(X).
# complex rule
animal(X) :- cat(X), dog(X), horse(X).
\end{lstlisting}

A $rule$ is defined with $<-$ or $:-$, which are the LARROW\_OPs. The name of the 
Rule is the $term$. That is, the Rule is referred to with the $term$.
The body of the Rule contains $or\_terms$. For evaluating a Rule to generate a Fact,
MIPL implements short circuit evaluation. If any of the constituent compounds in
$or\_terms$ evaluates to True, then MIPL stops and generates the Fact matching the
left side of the rule. To the left of the LARROW\_OP is the rule, which is a $term$. 
If the right of the rule operator is a Job, the left Term should be a dynamic Fact, 
as described above.
\medskip

\subsection{Terms}
\subsubsection{OR Term}
\begin{lstlisting}
or_terms
	: and_terms
	| or_terms `;' and_terms
	;
\end{lstlisting}
$or_terms$ contain either $and\_terms$ or $or\_terms$ separated by $;$. 
For the short circuit evaluation mentioned in rule evaluation, the $or\_terms$
is True if the left side of $;$ operator is True. The right side is evaluated
only if the left side is False.

\medskip

\subsubsection{AND Term}
\begin{lstlisting}
and_terms
	: term
	| and_terms `,' term	
	;
\end{lstlisting}
$and\_terms$ contain either $term$ or $and\_terms$ separated by $,$. 
For the short circuit evaluation mentioned in rule evaluation, the $and\_terms$
is False if the left side of $;$ operator is False. The right side is evaluated
only if the left side is True.
\medskip

\subsubsection{Term}
\begin{lstlisting}
term
	: IDENTIFIER		
	| IDENTIFIER `(' term_args `)'
	| IDENTIFIER `(' `*' `)'
	| REGEX			
	| REGEX `(' term_args `)'
	| REGEX `(' `*' `)'	
	| NOT term		
	| term_expr		
	| VARIABLE IS term	
	| term_expr `<' term_expr
	| term_expr `>' term_expr
	| term_expr LE_OP term_expr
	| term_expr GE_OP term_expr
	| term_expr EQ_OP term_expr
	| term_expr NE_OP term_expr
	;
\end{lstlisting}
A Term is a basic unit in MIPL. A Term can be used to define a fact with the
given IDENTIFIER.
\medskip

For defining a Query, Term can be used with argument lists, which are expressed by
$(\ term\_args\ )$. To query all data, Term can be used with an asterisk, which
is expressed by $*$. Also, Term can be a regular expression in MIPL; however,
Term as a regular expression can be only used in Query. All terms matched by the
regular expression will be queried with the given argument lists.
\medskip

By adding NOT, a Term can be negated. Also, Term Expressions can also be Terms.
In this case, Term contains the value of the computed results of Term Expressions.
As an extension of this, Terms and hence Term Expressions can be assigned to Variables,
by $VARIABLE\ IS\ term$.
\medskip

In addition, comparison operators with Term expressions are also Term. There are six
comparison operatros can be used in Term: $<, >, <=, >=, ==, !=$. LE\_OP refers $<=$,
GE\_OP refers $>=$, EQ\_OP refers $==$, and NE\_OP refers $!=$.
\medskip

%----------------------------------------------------------JINHYUNG}
%----------------------------------------------------------YOUNGHOON{

\subsection{Term Expression}
\begin{lstlisting}
term_expr
	: term_expr `+' term_fact
	| term_expr `-' term_fact
	| term_fact
	;
\end{lstlisting}

Term Expression is an expression which belongs to a Term.  Term 
Expressions can consist of binary operators `+' and `-' along with Term
Factors. These two operators left-associate Term Facts so that Term
Factors within a Term Expression can be computed with a left priority.

Term Expression, Term Factor, and Term Term are basic blocks that
consist of Term Expression.  So these three rules can be applied within
Term Expression, by the definition of MIPL grammar.
\medskip

\subsubsection{Term Factor}

\begin{lstlisting}
term_fact
	: term_fact `*' term_term
	| term_fact `/' term_term
	| term_term
	;
\end{lstlisting}

Term Factor is an expression which belongs to a Term Expression.  Term 
Factors can consist of binary operators `*' and `/' along with Term
Term.  These two operators left-associate Term Terms so that Term
Terms within a Term Factor can be computed with a left priority. Like,
mathematical notations, operators `*' and `/' have higher priority than
operators `+' and `-' since Term Facts are reduced 
\medskip


\subsubsection{Term Term}
\begin{lstlisting}
term_term
	: VARIABLE
	| NUMBER
	| `(' term_expr `)'
	;
\end{lstlisting}

Term Term is a terminal expression of Term Expression and its derivables,
belonging to a Term Factor.  Term Terms can consist of a Variable or a
Number, which serve as the basic blocks of Term Expressions.  Also, a Term
Expression can be reduced to a Term Term with a pair of parentheses, which
allows the operators, `-', `+', `*', and `/' can appear recursively in a
Term Expression.  This also means that `(' and `)' operators have higher
operation priority than other operators.
\medskip

\subsection{Term Arguments}

\begin{lstlisting}
term_args_cand
	: IDENTIFIER
	| IDENTIFIER `(' term_args `)'
	| VARIABLE
	| `_'
	| NUMBER
	| STRING_LITERAL
	;

term_args
	: term_args_cand
	| term_args `,' term_args_cand
	;

\end{lstlisting}

Term Argument Candidate is an Identifier, a Variable, an Anonymous
Variable, a Number, or a String.  An Identifier can have one
or more arguments within a pair of parentheses or no arguments.
This arguments again consist of Term Argument Candidates, which
allows a recursive form of argument definition.  For instance, 
$com(monitor, board(cpu, ram(4, 4)), P)$ is a valid form
of a Term declaration.

Term Argument List is used for Term definition. Hence, Term
Argument List can be a part of the declarations of Facts, Rules,
and Queries.  Sharing the same type of argument list for the
declarations of FRQs is possible because there are no type and
variable declarations in MIPL.  Thus, a Job definition and
a Job-calling Fact basically have the same argument list.
Also, MIPL's Job definitions allows variable numbers of return
values dynamically, the generated program
checks if the numbers of return values from Job and the numbers of
Fact names are matched.  Otherwise, a run-time error is generated.
\medskip

\subsection{Job}

\begin{lstlisting}
job
	: JOB IDENTIFIER `(' job_args `)' `{' stmt_list `}'
	;

job_args
	: job_args ',' VARIABLE
	| VARIABLE
	;
\end{lstlisting}

\label{Job_Section}
Job is MIPL's basic operation that provides matrix operations.
Job is a function-like task, with no nested definitions, no
overloading of definition, or generic types supported. Job uses
positional parameter passing with call-by-value.  A Job
definition starts with the keyword JOB. It denotes a number of variables and a set of Job
Statements.  At runtime, a compiled Job is distributed over
a MapReduce clusters, computed, and delivered to the user. For
example, when a user loads two matrices and calls a Job giving those
matrices, the matrices are split into several smaller matrices,
distributed to MapReduce clusters, and computed.

Job Arguments play a role in passing data from Facts to Job, by
delivering number values and converting Facts into matrices
so that the Job computes the number values and the matrices by
the algorithm described in the Job.
\medskip

\subsection{Job Statement}

\begin{lstlisting}
stmt
	: selection_stmt
	| compound_stmt
	| return_stmt
	| expr_stmt
	| iteration_stmt
	;
\end{lstlisting}

Job Statement is a sentence in a Job.  There are five types of
Job Statement: Job Selection Statement, Job Compound Statement,
Job Return Statement, Job Expression Statement, and Job Iteration
Statement.  Each of these will be explained in the following
sections.
\medskip

\subsubsection{Job Statement List}

\begin{lstlisting}
stmt_list
	: stmt
	| stmt_list stmt
	;
\end{lstlisting}

Job Statement List is a set of Job Statement, used in Job
declarations and Job Compound Statement.
\medskip


\subsubsection{Job Compound Statement}

\begin{lstlisting}
compound_stmt
	: `{' `}'
	| `{' stmt_list `}'
	;
\end{lstlisting}

Job Compound Statement is a Job Statement, which includes
zero or more Job Statements, surrounded by a pair of curly
brackets.
\medskip

\subsubsection{Job Return Statement}

\begin{lstlisting}
return_stmt
	: `@' expr `.'
	;

\end{lstlisting}

Job Return Statement is a sentence that returns a matrix.  Since
Job is likely executed on MapReduce clusters (unless configured
so or the matrix size is smaller than configured threshold), Job
Return Statement fetches the matrix.  The most interesting part
in MIPL with regard to Job Return Statement is a Job can contain
more than on Job Return Statement for returning values more than
one.  Accordingly, as a Job can contain a number of Job Selection
Statements and Job Iterative Statements, the combination of those
allows a Job to return variable numbers of return values. This is
possible because Job Return Statement does not ``return" immediately,
when its program counter meets Job Return Statement.  Instead, a
MIPL program keeps executing passing that Job Return Statement.
For example, below MIPL program returns three matrix values.  If
the given number of names for return values are not matched to the
number of values the actual Job returns, MIPL reports an error.

\begin{lstlisting}
I = 3;
[a, b, c] = job1(matrix1, matrix2, I).

job job1(A, B, NumRepeat) {
    while (NumRepeat > 0) {
       A *= B.
       @A.
       I = I - 1. 
    }
}
\end{lstlisting}

Because Job Return Statement does not ``return" immediately,
there's no acceptable Job Return Statement like $@.$, which some
languages provide to flow control such as ``return;" in C.  The
return operator $@$ is designed to deliver a matrix value from
MapReduce clusters to the user.  When a user want to quit the Job
in the mid of the Job, the user should use Job Control Statements
such as If Statement to do so.
\medskip


\subsubsection{Job Expression Statement}

\begin{lstlisting}
expr_stmt
	: expr `.'
	;
\end{lstlisting}

Job Expression Statement is a Job Statement that includes a Job
Expression.  This allows users use Job Expressions in a Job.
\medskip

\subsubsection{Job Selection Statement}

\begin{lstlisting}
selection_stmt
	: IF `(' bool_expr `)' stmt
	| IF `(' bool_expr `)' stmt ELSE stmt
	;
\end{lstlisting}

Job Selection Statement is a conditional control Job Statement,
which checks the value of given Job Boolean Expression and executes
the statements accordingly.  If the value of the Job Expression is
true, the first Job Statement will be executed.  If the value of
the expression is false and the ELSE statement is provided, this
statement will be executed.  The IF ELSE syntax is ambiguous,
however, by Yacc's default behavior, reduce precedes shift, which
results in that the ELSE is associated with closer IF.
\medskip

\subsubsection{Job Iteration Statement}

\begin{lstlisting}
iteration_stmt
	: WHILE `(' bool_expr `)' stmt
	| DO stmt WHILE `(' bool_expr `)' `.'
	;
\end{lstlisting}

Job Iteration Statement is a statement, that repeatedly executes
the given Job Statement according to the value of Job Expression.

WHILE statement executes the given Job Statement while the Job
Boolean Expression's value is true.  DO WHILE statement executes the
Job Statement once first, and then executes the Job Statement repeatedly
while the Job Boolean Expression's value is true.
\medskip


\subsection{Job Expression}

\begin{lstlisting}
expr
	: additive_expr
	| VARIABLE assign_op assign_expr
	;
\end{lstlisting}

\label{Expression_Section}
Job Expression is a grammatical unit that conveys a numeric value,
which includes a matrix or a number.  Whether a Job Expression
contains a matrix or a number is decided dynamically at runtime,
as well as how big is the matrix, since MIPL is dynamic typed.

Job Expressions assigns a numberic value to a Variable contained
at the left-hand side.  The value of Job Expression is the value
newly assigned to the Variable.  This may relays multiple times,
for example, $Variable1 = Variable2 = Variable3 = 3.$.
\medskip


\subsubsection{Job Assignment Operator}

\begin{lstlisting}
assign_op
	: `='
	| LARROW_OP
	| MUL_ASSIGN
	| DIV_ASSIGN
	| MOD_ASSIGN
	| ADD_ASSIGN
	| SUB_ASSIGN
	;
\end{lstlisting}

Job Assignment Operator has seven types.  1. `=' and ``$<-$" are used
for Assignment.  2. ``*=" is Muliplication Assignment Operator.  3. 
``/=" is Division Assignment Operator.  4. ``\%=" is Modular Assignment
Operator.  5. ``+=" is Addition Assignment Operator.  6. ``-=" is
Subtraction Assignment Operator.  Each Assignment Operators computes
the value of the operation using the left-hand side value and the
right-hand side value and assigns the value to the left-hand side
Variable.
\medskip


%----------------------------------------------------------YOUNGHOON}

%----------------------------------------------------------AKSHAI{
\subsubsection{Job Additive Expression}

The character sequence `+' and `-' mean addition and subtraction, 
respectively. These operators in Additive expressions, or additive\_expr, 
also associate left to right and have lower precedence than the 
multiplicative expressions, termed multiplicative\_expr, described next.

\begin{lstlisting}
additive_expr
	: multiplicative_expr
	| additive_expr `+' multiplicative_expr
	| additive_expr `-' multiplicative_expr
	;
\end{lstlisting}

Given two multiplicative expressions, connected by an additive operator, 
the expression evaluates to the result of applying the operator to the
operands. However, this depends on the type of the operands, which can
range from integer or doubles to matrices. In the case of the latter,
the evaluation of the expression results in a matrix, formed by the 
application of the operation on the multiplicative expressions.
The `+' and the `-' operators can be applied to numerical and matrix types 
and errors are generated for incompatible types.
\medskip

\subsubsection{Job Multiplicative Expression}

The character sequence `*' and `/' mean multiplication and division, 
respectively. The sequence `\%' represents the modulus operation. That
is, the remainder after the division of the right operand into the left
operand. These operators in Multiplicative expressions, or 
multiplicative\_expr, associate left to right and have lower precedence 
than unary expressions.

\begin{lstlisting}
multiplicative_expr
	: unary_expr
	| multiplicative_expr `*' unary_expr
	| multiplicative_expr `/' unary_expr
	| multiplicative_expr `%' unary_expr
	;
\end{lstlisting}

As with the additive expressions, the operators are applicable to types
where the mathematical equivalent is valid. For the case of multiplying
matrices, the requirement that the number of columns of the left operand 
be equal to the number of rows of the right operand. The result of the
evaluation of the operation is contained within the set of basic data 
types. The multiplication of a matrix expression with another results in 
a matrix expression. As with the additive expressions, errors may be 
generated at run-time for incompatible operations. For instance, the
multiplication of two matrices with incompatible number of rows and
columns will result in an error.
\medskip

\subsubsection{Job Unary Expression}
  
A unary expression contains the unary operators `+' for positive and `-' 
for negative. The `+' operator when applied to an arithmetic type does
not change the value of the type. The `-' operator when applied to an
arithmetic type, produces the negative of its operand. Similarly, for
matrices, the '+' does not change the contents of the matrix, while
'-' negates the value of the contents of the matrix.


\begin{lstlisting}
unary_expr
	: postfix_expr
	| `+' unary_expr
	| `-' unary_expr
	;
\end{lstlisting}
\medskip

\subsubsection{Job PostFix Expression}

A postfix expression is either a primary expression, an matrix access or 
a Job call.

\begin{lstlisting}
postfix_expr
	: primary_expr
	| VARIABLE `[' array_idx_list `]' `[' array_idx_list `]'
	| IDENTIFIER `(' `)'
	| IDENTIFIER `(' nested_jobcall_args `)'
	;
\end{lstlisting}
\label{Job_PostFix_Expression}

A VARIABLE is like an IDENTIFIER but it begins with an uppercase letter,
followed by lowercase or uppercase alphabets or underscore. A VARIABLE
followed by two `[' and `]' is a matrix access. The matrix is accessed
by subscripts, also known as the array index list, described in a later
section. The subscripts correspond exactly to the two dimensional 
interpretation of the matrix, where the the subscript within the first 
set of square brackets refers to the row of the matrix while the second
subscript refers to the corresponding pressure. The VARIABLE must refer 
to a matrix type. Using array index lists, any subset of the matrix can be
obtained. The result of this postfix expression can be a single value of
the type of the matrix or any subset of the matrix.

An IDENTIFIER followed by a set of paranthesis `(' and `)' is a Job
invocation. A Job is defined according to \ref{Job_Section}. Only a defined
Job may be invoked using the IDENTIFIER referring to the Job name,
followed by `(' and `)', where the parantheses enclose arguments passed
to the Job. If there were no arguments defined for the Job, the
parantheses do not include anything and if there were arguments defined, 
arguments that match the number of arguments in the Job definition, must
be provided. These arguments are provided using the argument\_expr\_list 
defined below. The arguments passed are by value. That is, the arguments 
are copied and the copies are passed to the Job being invoked. Special
care is taken with Matrices as they can be very large. MIPL handles 
matrices in a lazy fashion. Using a built-in Job to "load" a matrix, 
does not load it immediately and is kept around as a filename till it is
accessed.
\medskip

\subsubsection{Job Primary Expression}

A Job primary expression is the most basic type of expression. In any
expression, a primary expression has the highest precedence.

\begin{lstlisting}
primary_expr
	: IDENTIFIER
	| VARIABLE
	| NUMBER
	| `(' expr `)'
	;
\end{lstlisting}

An IDENTIFIER is a primary expression. An IDENTIFIER need not be declared
as MIPL is dynamically and weakly typed. Any token matching the description
of an IDENTIFIER, starting with a lowercase letter followed by uppercase and
underscore characters, is recognized as one. An IDENTIFIER can be referred to
in Terms.

A VARIABLE is also a primary expression. A VARIABLE is used to denote 
data types with location values such as matrices in order to distinguish them 
from IDENTIFIERS.

A NUMBER refers to a the decimal representation of a number and is also a 
primary expression. Scientific notation is also supported, following the 
E-notation. $10^{x}$ in scientific notation is simply written with either an
`e' or an `E' followed by x.

Finally, an expression enclosed with parantheses is also a primary expression 
and the value of this expression is the same as if the parantheses did not 
exist.
\medskip

\subsubsection{Job Array Index List}

An array index list can be used to access the relevant portions of a matrix. 
The relevant portions of a matrix are referred to using array index elements, 
described below. The array index elements are evaluated left to right.

\begin{lstlisting}
array_idx_list
	: array_idx_elmt
	| array_idx_list `,' array_idx_elmt
	;
\end{lstlisting}

Using the list of array indices, any relevant, valid portion of a matrix may
be accessed. For instance, given a 5 x 7 matrix A, A[2$\sim$4,6$\sim$][1] produces the
subset of the matrix A, a 5 x 1 result, consisting of its third, fourth, 
fifth and seventh rows and its second column.
\medskip

\subsubsection{Job Array Index Element}

An array index element is used to access a matrix. It consists of either 
the character `$\sim$' representing all valid indices or any combination of placing
a NUMBER to either side of the character. 

\begin{lstlisting}
array_idx_elmt
	: `\~'
	| `~' NUMBER
	| NUMBER
	| NUMBER `~'
	| NUMBER `~' NUMBER
	;
\end{lstlisting}

NUMBER here refers to any integer that denotes a valid index for the 
matrix, referring to either its $i^{th}$ row or column. Rows and columns are 
referred to starting at 0. If a NUMBER is not specified at either 
the left or the right of the `$\sim$' character, it defaults to the extreme 
of the matrix subscript. In other words, the smallest valid subscript in 
the case of the left, which is 0, and the largest valid subscript in the 
case of the right, which is the total number of rows or columns depending
on, which is being accessed.
\medskip

\subsubsection{Job Argument Expression List}

An argument expression list represents the various arguments that may be
passed to a Job invocation. 

\begin{lstlisting}
nested_jobcall_args
	: expr
	| nested_jobcall_args `,' expr
	;
\end{lstlisting}

These arguments are evaluated left to right and can include any assignment 
expression as described above in \ref{Expression_Section}.
\medskip

\subsection{Job Boolean Expression}

Job Boolean Expression is a basic Expression that can be evaluated as
true or false.  Job Selection Statement and Job Iteration Statement
contains a Job Boolean Expression in order to determine whether to proceed
or how many times to repeat.  Unlike C Programming Language, MIPL strictly
distinguish Job Boolean Expressions from Job Expressions as Job Boolean
Expressions can be used only for conditions in Job Selection Statements
and Job Iteration Statements.  Thus, a Job Boolean Expression cannot be
stored in a variable or cannot be used as a statement.

The character sequence OR\_OP refers to the sequence ``\textbardbl", which  
denotes the logical OR operator. The logical AND expressions that are 
separated by the logical OR operator associate left. The next section 
details how the logical AND expressions, logical\_and\_expr, which have 
higher precedence than the logical OR expression, are evaluated. 

\begin{lstlisting}
bool_expr
	: logical_and_expr
	| bool_expr OR_OP logical_and_expr
	;
\end{lstlisting}

Given a binary OR\_OP, the Job Boolean Expression evaluates to true 
if either of the nested bool\_expr or the logical\_and\_expr evaluates
to true, and evaluates to false if both of the bool\_expr or
logical\_and\_expr evaluate to false, The order of evaluation is left
to right, also known as the ``short circuit" evaluation, whereby
the logical\_and\_expr on the right of the OR\_OP is evaluated if
and only if the logical\_and\_expr on the left of the OR\_OP 
evaluated to false. 
\medskip

\subsubsection{Job Logical AND Expression}

The character sequence AND\_OP refers to the sequence ``\&\&", which 
denotes the logical AND operator. The equality expressions that are 
separated by the logical OR operator associate left. The next section 
details how equality expressions, equality\_expr, which have higher 
precedence than the logical OR expression, are evaluated. 

\begin{lstlisting}
logical_and_expr
	: equality_expr
	| logical_and_expr AND_OP equality_expr
	;
\end{lstlisting}

Given a binary AND\_OP, the Logical AND expression evaluates to true,
if both of the logical\_and\_expr evaluate to true and evaluates to false,
if either of the logical\_and\_expr evaluate to false. 
The order of evaluation is again left to right, whereby the 
equality\_expr on the right of the AND\_OP is evaluated if and only 
if the equality\_expr on the left of the OR\_OP evaluated to true.
\medskip

\subsubsection{Job Equality Expression}

The character sequence EQ\_OP refers to the sequence ``==" and the 
character sequence NE\_OP refers to the sequence ``!=". These operators, 
equal to and not equal to, respectively, cause the expression to 
evaluate to false and true.  These operators that separate the
relational expressions, relational\_expr, associate left. The
operators EQ\_OP and NE\_OP have lower precedence than the 
operators that constitute the relational expressions, described in the 
next section. 

\begin{lstlisting}
equality_expr
	: relational_expr
	| expr EQ_OP expr
	| expr NE_OP expr
	| equality_expr EQ_OP relational_expr
	| equality_expr NE_OP relational_expr
	;
\end{lstlisting}

Given two relational expressions, connected by a EQ\_OP or a NE\_OP, 
Job Expression or Job Equality Expression evaluates to true, 
if and only if the left and the right Job Expressions both evaluate
to the same value in the case of a EQ\_OP and to different values in
the case of a NE\_OP.  Otherwise, the equality expression evaluates
to false.
\medskip

\subsubsection{Job Relational Expression}

The character sequence LE\_OP refers to the sequence ``$<=$", for less 
than or equal to and GE\_OP refers to the sequence ``$>=$", for greater 
than or equal to. The sequences `$<$' and `$>$' mean exactly the same as 
their mathematical counterpart relations. These operators also associate 
left to right and have lower precedence than the operators in the additive 
expression type, additive\_expr, described in the following section. The 
evaluation of a relational expression is once again a truth value, 
represented by the false and true. 

\begin{lstlisting}
relational_expr
	: boolvalue_expr
	| expr `<' expr
	| expr `>' expr
	| expr LE_OP expr
	| expr GE_OP expr
	;
\end{lstlisting}

Given two Job Expressions, connected by the above mentioned operators, 
Job Relational Expression evalutes to true if the mathematical relation
represented by the operation is satisfied. Otherwise, Job Relational
Expression evaluates to false. 
\medskip


\subsubsection{Job BooleanValue Expression}

\begin{lstlisting}
boolvalue_expr
	: TRUE
	| FALSE
	| '(' bool_expr ')'
	;
\end{lstlisting}

The character sequences TRUE and FALSE refer to the sequences ``true"
and ``false" respectively.  Job BooleanValue Expression can be expanded
to a nested Job Boolean Expression embraced with a pare of parentheses.
\medskip


%----------------------------------------------------------AKSHAI}
%----------------------------------------------------------DANIEL}
\section{Job Operators}
MIPL has various Job Operators to support matrix operations in a Job
description. Although, some of these were discussed in the Syntax
section above, we reiterate them in the context of matrices and for
completion. Some of the operators discussed below are not added to 
the grammar. We propose them here are as very simply modifications 
are needed to the grammar to incorporate them.

\subsection{Operator $+$}
Addition or unary plus. A+B adds A and B. A and B must have the same size,
unless one is a number. A number can be added to a matrix of any size.
\medskip

\subsection{Operator $-$}
Subtraction or unary minus. A-B subtracts B from A. A and B must have the
same size, unless one is a number. A number can be subtracted from a matrix
of any size.
\medskip

\subsection{Operator $*$}
Matrix multiplication. C <- A $*$ B is the linear algebraic product of the
matrices A and B. More precisely,

$C(i,j) <- \displaystyle\sum\limits_{k=1}^n A(i,k)B(k,j).$

For matrices A and B, the number of columns of A must equal the number of
rows of B. A number can multiply a matrix of any size.
\medskip

\subsection{Operator $.*$}
Cell-wise matrix multiplication. A.*B is the element-by-element product
of the matrices A and B. A and B must have the same size, unless one of
them is a number.
\medskip

\subsection{Operator $/$}
Matrix division. Given A/B, B should be a number while A can be either a
number or a matrix.
\medskip

\subsection{Operator $./$}
Cell-wise Matrix division. A./B is the matrix with elements A(i,j)/B(i,j).
A and B must have the same size, unless one of them is a number.
\medskip

\subsection{Operator \^{}}
Matrix power. X\^{}p is X to the power p, if p is a integer. The power
is computed by repeated multiplication.
\medskip

\subsection{Operator .\^{}}
Cell-wise matrix power. A .\^{} B is the matrix with elements A(i,j) to
the B(i,j) power. A and B must have the same size, unless one of them
is a number.

\section{Built-in Matrices}
Built-in Matrices are special type of matrices that can be used for
various purposes.
\medskip

\subsection{zeros}
Keyword zeros creates a matrix of all zeros.
\begin{lstlisting}
Y <- zeros.
\end{lstlisting}

$Y <- zeros.$ returns a matrix of zeros.
\medskip

\subsection{ones}
Keyword ones creates a matrix of all ones.
\begin{lstlisting}
Y <- ones.
\end{lstlisting}

$Y <- ones.$ returns a matrix of ones.
\medskip

\subsection{eye}
Keyword eye returns the identity matrix.
\begin{lstlisting}
Y <- eye.
\end{lstlisting}

$Y <- eye.$ returns a matrix with 1's on the diagonal and 0's elsewhere.
\medskip

\subsection{rand}
Keyword rand generates a matrix of random numbers whose elements are
uniformly distributed in the interval (0,1).
\begin{lstlisting}
Y <- rand.
rand
\end{lstlisting}

Y <- rand. returns a matrix of random entries. Note that a random value
is generated when an element of rand is accessed. For example,
$B <- rand + A. C <- rand + A.$ results in two different matrices B and A.
In order to get a constant random matrix, a Job rand() can be used.
$R <- rand(3,4). B <- R + A. C <- R + A.$ results in the same two matrices B
and A.


\section{Built-in Jobs}
There are several Built-in Jobs, while MIPL provides an easy way
to add a new Job.
\medskip

\subsection{zeros()}
Keyword zeros(n, m) creates a n-by-m matrix of all zeros.
\begin{lstlisting}
Y <- zeros(10, 20).
\end{lstlisting}

$Y <- zeros(10, 20).$ returns a 10-by-20 matrix of zeros.
\medskip

\subsection{ones()}
Keyword ones(n, m) creates a n-by-m matrix of all ones.
\begin{lstlisting}
Y <- ones(10, 20).
\end{lstlisting}

$Y <- ones(10, 20).$ returns a 10-by-20 matrix of ones.
Job ones() is defined as the following:
\begin{lstlisting}
job ones(N, M) {
    @zeros(N, M) + 1.
}
\end{lstlisting}
\medskip

\subsection{eye()}
Keyword eye(n) returns the identity n-by-n matrix.
\begin{lstlisting}
Y <- eye.
\end{lstlisting}

$Y <- eye(10).$ returns a 10-by-10 matrix with 1's on the diagonal and 0's elsewhere.
Job eye() is defined as the following:
\begin{lstlisting}
job eye(N) {
    @zeros(N, N) + eye.
} 
\end{lstlisting}
\medskip

\subsection{rand()}
Keyword rand(n, m) generates a n-by-m matrix of random numbers whose elements are
uniformly distributed in the interval (0,1).
\begin{lstlisting}
Y <- rand(10, 20).
rand
\end{lstlisting}

Y <- rand(10, 20). returns a 10-by-20 matrix of random entries.

Job rand() is defined as the following:
\begin{lstlisting}
job rand(N, M) {
    @zeros(N, M) + rand.
} 
\end{lstlisting}
\medskip

\subsection{bind()}
Job bind() concatenates matrices horizontally.
\begin{lstlisting}
C <- bind(A,B).
\end{lstlisting}

$C <- bind(A,B).$ concatenates the matrices A and B horizontally.
\medskip

\subsection{vbind()}
Job vbind() concatenates matrices vertically.
\begin{lstlisting}
C <- vbind(A,B).
\end{lstlisting}

$C <- vbind(A,B).$ concatenates the matrices A and B vertically.
\medskip

\subsection{eig()}
Job eig() returns a vector of the eigenvalues of matrix.
\begin{lstlisting}
D <- eig(A).
\end{lstlisting}

$d <- eig(A)$ returns a vector of the eigenvalues of matrix A.
\medskip

\subsection{max()}
Job max() returns the largest elements of a matrix.
\begin{lstlisting}
C <- max(A).
\end{lstlisting}

$D <- max(A).$ returns the largest element in A.
\medskip

\subsection{mean()}
Job mean() returns the mean values of the elements of a matrix.
\begin{lstlisting}
M <- mean(A).
\end{lstlisting}

$D <- mean(A).$ returns the mean value of A.
\medskip

\subsection{min()}
Job min() returns the smallest elements of a matrix.
\begin{lstlisting}
C <- min(A).
\end{lstlisting}

$D <- min(A).$ returns the smallest element in A.
\medskip

\subsection{sum()}
Job sum() returns sums of a matrix.
\begin{lstlisting}
B <- sum(A).
\end{lstlisting}

$D <- sum(A).$ returns the sum of the elements.
\medskip

\subsection{abs()}
Job abs() returns the absolute value.
\begin{lstlisting}
Y <- abs(X).
\end{lstlisting}

$D <- abs(X).$ returns the absolute value, X, for each element of X.
\medskip

\bigskip
\bigskip
\bigskip
\bigskip

\section*{Appendix - Grammar}

MIPL's full grammar is here. MIPL's Yacc grammar has only one
shift/reduce conflict, which C language also has, without using
operator associativity declarations. This grammar does not 
include some of the additional operators and the built-in matrices
and Jobs. They can, however, be easily incorporated.

\lstinputlisting{../../build/mipl.grammar}
%\input{../../build/mipl_grammar.tex}
%\lstinputlisting{tablified_grammar}

% Appendix
%\appendix
%\section*{APPENDIX}
%\setcounter{section}{1}
%In this appendix, we measure

%\appendixhead{ZHOU}

% Acknowledgments
%\begin{acks}
%The authors would like to thank Dr. Maura Turolla of Telecom
%\end{acks}

% Bibliography
%\bibliographystyle{acmsmall}
%\bibliography{reference_manual_ref}

% History dates
%\received{February 2007}{March 2009}{June 2009}

% Electronic Appendix
%\elecappendix

\medskip

%\section{This is an example of Appendix section head}

\end{document}
